\documentclass[12pt,a4paper,oneside]{report}
\usepackage[utf8]{inputenc}
\usepackage{amsmath}
\usepackage{amsfonts}
\usepackage{amssymb}
\usepackage{graphicx}
\begin{document}
\title{selfstudy 3 - peer review}
\author{sw608f14 - Søren Skibsted Als, Lars Andersen, \\Mathias Winde Pedersen \& Daniel Steinar Fridjonsson}
\maketitle
\section*{Exercise 1 - 25 points}
You did not use chen notation. \textit{-10\%}
By your diagram, a BookCopy can only be borrowed once throughout its existence. \textit{-5 \%}
You give opportunity to have a copy of a book without a copy of relation.\textit{-5\%}

So you get \textit{80\%}.
\section*{Exercise 2 - 15 points}
You made a relational diagram, but you should just provide relations. \textit{-50\%} as correct notation is key for this course.
You made Statement wrong as it is a weak entity, so it needs a foreign key for its primary key. \textit{-10\%}

So you get \textit{40\%} as the relations are mostly correct, but is represented as a diagram which they should not.

\section*{Exercise 3 - 15 points}
You get \textit{100\%}.

\section*{Exercise 4 - 30 points}
\subsection*{1.}
Domain part is wrong in the Catalog part. So \textit{-15\%} as half of domain part is wrong.

You get \textit{85\%}
\subsection*{2.}
Tuple calculus you missed parenthesis and is thus wrong, as \textit{and} takes precedence over \textit{or}. \textit{-10\%}

You get \textit{90\%}
\subsection*{3.}
wrong in domain part. $<i,x,y> \in Parts$ should be $<j,x,y> \in Parts$. \textit{-10\%}.

You get \textit{90\%}

\subsection*{4.}
Cost is not in parts but in catalog.
Curly bracket ending is misplaced.
Your pid and sid comparisons are wrong. pids should be equal, and sids shot not be equal.
So for relational you lose \textit{33\%}.

For tuple $s_1$ is not specified. Cost from parts also wrong here.
pid and sid comparison also wrong here.
You lose \textit{33\%}.

Domain is almost correct, you miss a $cost'$. You lose \textit{3\%}

So you get \textit{30\%} in total for this. 

\subsection*{5.}
You get \textit{100\%}


In total for the whole exercise you get $\frac{85 + 90 + 90 + 30 + 100}{5}\% = 79\%$

\section*{Exercise 5 - 15 points}
You get \textit{100\%}.

\section*{Grading}
In total you get $(0.8*25 + 0.4*15 + 15*1 + 30*0.79 + 15*1)\%= 79.7\%$.
So you get a grade of 7.
\end{document}