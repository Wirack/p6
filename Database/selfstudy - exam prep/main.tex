\documentclass[10pt,a4paper,final]{report}
\usepackage[utf8]{inputenc}
\usepackage[english]{babel}
\usepackage{amsmath}
\usepackage{amsfonts}
\usepackage{amssymb}
\usepackage{float}
\usepackage{enumitem}
\usepackage{graphicx}
\usepackage{listings}


\title{Selfstudy 7 - Database}
\author{sw608f14 - Daniel S. F., Lars A., Mathias W. P. and Søren S. A.}

\begin{document}

\maketitle

\section*{1. Multiple Choice}
The marked answers will be presented in lists.
\subsection*{1.}
\begin{itemize}
     \item The results of both queries are sometimes the same, sometimes not; depending on the
relations’ extension
\end{itemize}

\subsection*{2.}
\begin{itemize}
     \item The query result is NULL
\end{itemize}

\subsection*{3.}
\begin{itemize}
     \item is equivalent to the cross product
\end{itemize}

\subsection*{4.}
\begin{itemize}
     \item SELECT a,b FROM R GROUP BY a,b;
     \item SELECT a FROM R GROUP BY a,b;
\end{itemize}


\subsection*{5.}
\begin{itemize}
     \item SELECT a FROM R JOIN S ON a=b;
\end{itemize}


\subsection*{6.}
\begin{itemize}
     \item contains any prime attributes
     \item $\alpha$ is a super key
\end{itemize}

\section*{2 -- Integrity Constraints}
\begin{lstlisting}[language=SQL]
CREATE TABLE user (
ID INTEGER PRIMARY KEY,
name VARCHAR(32) UNIQUE,
age INTEGER,
CHECK(age >= 16 AND age < 150)
);

CREATE TABLE topic (
ID INTEGER PRIMARY KEY,
name VARCHAR(32),
author INTEGER,
FOREIGN KEY(author) REFERENCES user(ID) ON DELETE CASCADE
);

CREATE TABLE post (
ID INTEGER PRIMARY KEY,
author INTEGER,
topic INTEGER,
subject VARCHAR(32),
content VARCHAR(1024),
FOREIGN KEY(author) REFERENCES user(ID) ON DELETE CASCADE,
FOREIGN KEY(topic) REFERENCES topic(ID) ON DELETE CASCADE
);
\end{lstlisting}

\section*{3 -- Recursion in SQL}
\begin{lstlisting}[language=SQL]
WITH RECURSIVE tempTable(number, counter) AS (
(SELECT 2,5)
UNION
SELECT (number*2), counter - 1
FROM tempTable
WHERE counter > 1
)
SELECT max(number) FROM tempTable
\end{lstlisting}

\section{4 -- SQL Queries}

\subsection*{1.}
\begin{lstlisting}[language=SQL]
SELECT creator, count(BID) LIMIT 10
FROM post 
GROUP BY creator
ORDER BY count(BID) DESC
\end{lstlisting}

\subsection*{2.}
\begin{lstlisting}[language=SQL]
SELECT title, count(BID)
FROM topic JOIN post ON topic.TID = post.TID
GROUP BY TID
HAVING count(BID) = max(count(BID))
\end{lstlisting}

\subsection*{3.}
\begin{lstlisting}[language=SQL]
SELECT title LIMIT 10
FROM topic JOIN post ON topic.TID = post.TID
ORDER BY creationDate DESC
\end{lstlisting}

\subsection*{4.}
\begin{lstlisting}[language=SQL]
SELECT creator as UID
FROM post
GROUP BY creator
HAVING count(BID) > 30
\end{lstlisting}

\section*{5 -- Normalization}
\subsection*{1.}
Candidate keys:
AC, BC, AD, BD

It fulfills 2NF because all attributes are prime.
For that reason it is also 3NF, on third rule.
That gives us that it is not BCNF since it is the third rule that makes in 3NF.

\subsection*{2.}
Candidate keys:
EC, EAB

It fulfills 2NF as all non-prime attribute is fully functionally dependent on each candidate key.

It is not 3NF as $AB \rightarrow CDF$ does not fulfill any of the rules.

\subsection*{3.}
Candidate keys:
A, B, C, D

2NF as all left sides are candidate keys, giving fully functional dependency.

3NF and BCNF as all left sides are super keys (second rule fulfilled)

\section*{6 -- Functional Dependencies}
\subsection*{1.}

\begin{itemize}
     \item[$AB \rightarrow C$] as the last two tuples break it
     \item[$CDE \rightarrow B$] true
     \item[$CE \rightarrow A$] true
     \item[$BC \rightarrow E$] no, tuple 3 and 4 breaks it.
     \item[$BCDE \rightarrow A$] true
     \item[$D \rightarrow AE$] nope, tuple 2 and 3 breakts it as an example.
\end{itemize}

\subsection*{2.}
BCDE is a super key, as CD is a candidate key.\\
ABE is a candidate key.\\
DE is not a super key.\\
CD is a candidate key.
\end{document}