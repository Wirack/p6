\chapter{Product Backlog}\label{app:further-deflopment}
\begin{description}[style=nextline]
\item[Editing pictograms]
It should be possible to edit pictograms, when it is the same author editing it. 
By editing we mean a change to either its name, sound, picture, access settings, or tags.
Currently, if you have to edit a pictogram, you load it into \textit{Piktotegner} from \textit{Pictosearch}, and save it as another instance in the database.

\item[Crop camera pictures]
A request from a customer was to crop pictures taken with the camera.
For example if the user takes a picture of an apple on a table and would like to only have the apple, they would like to exclude the unnecessary parts of the picture before inserting it.

\item[Undo button]
Some colleagues suggested to add an undo button, as it was tedious to select the entity first and then delete it every time they made a mistake.

\item[Categorising pictograms]
Newly created pictograms could have the possibility of being categorised when saved.

\item[Change colour of drawn entities]
It should be possible to change the colour of already drawn entities.
This could be by either drag-and-dropping a colour onto an entity or selecting the entity and then choosing a colour.

\item[Camera dialogue buttons]
During the usability test, the testees had trouble reaching the \textit{capture} button when they tried to take a picture.
A suggestion is to reorder the buttons such that they are on the side of the tablet.

\item[Template pictures]
A suggestion was to add a setup assistant for template pictures, such that users could add some customisable figures into their pictogram.
An example of this could be a stick figure, but with a customisable posture. 

\item[Memory issues]
The application uses a lot of high quality pictures which causes memory issues on certain tablets. 
These are the pictures seen in the help dialogue and the camera pictures.

\item[New icon for clearing the canvas]
The current icon for clearing the canvas is a trash bin with the text \textit{Ny} which means new. 
The text does not hint at the functionality usually associated with such an icon, and should thus be changed to something more appropriate.
A suggestion to such an icon would be the usual blank paper with a folded edge and a plus sign, as seen in most text editors and graphics painting application. 

\item[Close and exit icons]
During the usability test, the testees stated that the citizens associate the current close and exit icons with cancelled activities.
Therefore, these icons should be changed or removed.

\item[Flatten button icon]
The flatten button, which puts entities to the back of the canvas, was not clearly understood by the testees during the usability test and should therefore be redesigned.

\item[Pinching]
During the usability test, testees tried to resize an entity by pinching it on the touchscreen. 
This functionality is not currently supported, but since people are used to this gesture when using touch pad applications, this functionality should be considered.

\item[Names in save dialogue]
The text fields in the save dialogue were causing confusion both to the customers and our colleagues.
The idea of these fields was that the inline text was the text written on the pictogram whereas the name was the actual name of the pictogram which the guardian used as reference.
An example of this is two pictures, one with a red car and the other with a blue car, having the same inline text ``Car'', but for reference the guardian would like to know the colour of the car through the name, thus ``Red car'' and ``Blue car''.

\item[Rethink preview button]
Both in the usability tests and the sprint end meeting, the preview button was found unintuitive.
The doubt was first the double functionality of the button, as it is a preview button and a swap colour button.
A solution could be to scrap the preview button and instead have a clearer illustration of what tool is selected, and at the same time have a new button with its functionality being the swapping of colours.

\item[Alternative help dialogue]
During the sprint 4 sprint end meeting it was proposed that the help pictures of the help dialogue should be made larger.
As an alternative to this, it was proposed to provide short video clips to demonstrate how to use the application, and is the suggestion we recommend for future groups.
If the help pictures solution is kept, clarifying text for these pictures would be beneficial.

\item[Placement of timer]
The application \textit{Timer}, developed by another \textit{GIRAF} group, provides a timer to keep track on how long a citizen is provided for a given activity. 
The timer is placed in the top right corner of the tablet and should be taken into account, as at this point in time it overlays the exit button, or the \textit{Timer} application should add a functionality to move this timer.

\item[Reconsider selection of entities]
During the usability tests, it was evident that the testees found the selection tool unintuitive.
With training they may get used to the tool, however, they tried some other gestures to select an entity and is worth considering.
Gestures used was double click on a given entity to select it or long clicking.
Maybe these forms of gestures should replace the selection tool or be an alternative way to select entities.

\item[Saving and loading \textit{drawStack} with bitmaps in the database]
Find a reasonably efficient way to serialize bitmaps, so they can be stored in the \textit{drawStack} in the database and loaded again.

\end{description}