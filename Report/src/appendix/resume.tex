\chapter{Resume}
We develop an application for creating pictograms, \textit{Piktotegner}.
\textit{Piktotegner} was first developed in the spring of 2013.
In the spring of 2014, we have further developed \textit{Piktotegner} to be usable in the larger software package called \textit{GIRAF}.
You are taken through the development process of \textit{Piktotegner}, from the application not being deployable, to an application that is usable for the \textit{GIRAF} costumers.
In the current version of \textit{Piktotegner} you are able to draw lines, ellipses, and rectangles.
Additionally, you are also able to take pictures with ones tablet camera, record sound or use an auto generated one, and reuse other pictograms from the \textit{GIRAF} database.
Regarding the database, it is possible to load and save the created pictograms, such that they can be used by the other applications of \textit{GIRAF}.

\textit{GIRAF} is an autism communication software package developed for Android tablets by students at Aalborg University.
\textit{GIRAF} is meant to aid autistic people and their guardians in communication and learning. 
In the package you find applications to construct and read sentences by help of pictograms, speech training tools, categorisation tools, scheduling applications, and more.

The development of \textit{GIRAF} was done in a multi project setting, as the applications of \textit{GIRAF} are dependent on each other as well as the need to have a common design. 
To collaborate in the multi project setting, a SCRUM of SCRUMs is applied, and a description thereof can be found in the report.
In the description, you find how the technique has been applied in the multi project setting as well as the outcome and the experience gained thereof.

In the report you find a description of how we handle geometric problems of drawn objects, such as rotation, resizing, and selection.
You can also find our thought process of designing an intuitive user interface in collaboration with the customers.
In that regard, usability tests were performed to evaluate this interface design.
A description of the flaws found in \textit{Piktotegner} can therefore also be found, and can be used for future development.
Furthermore, collaboration with other groups to develop a shared audio library and how we utilised a code review is described.
The shared audio library includes a media player and a Text-To-Speech method.

The report is structured in a chapter containing an analysis of the organisational context, four sprint chapters, a chapter with our reflections of the multi project and our development process, as well as two collaboration chapters.
The sprint chapters include the sprint backlog for that sprint and a description of the solutions used for each issue of the sprint backlog.
The two collaboration chapters are written in collaboration with other groups.