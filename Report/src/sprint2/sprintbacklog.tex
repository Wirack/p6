\section{Sprint Backlog}
For sprint 2, the sprint backlog consists of the following.

\begin{description}[style=nextline]
\item[Load existing audiofile to the record fragment]
After recording an audio file and exiting the record fragment, the recorded file can not be played when you enter the record fragment again, which it should.

\item[playButton press before no recording is performed]
The play button should be deactivated when no recording exists.

%\item[Save without name] 
%When trying to save a picture without name, it just gave it the date time. Should not be allowed to save like that.

%\item[Change name Croc to PictoCreator]
%Make contact with different groups to get croc renamed to PictoCreator everywhere.

\item[Change PictoCreator icon]
Needs to be changed to a pencil drawing on a paper, as requested by the customer.

\item[Record dialogue GUI change]
Needs to be changed to one record button that switches icon to a stop icon when recording.
In addition have a single toggle button for playing and stopping of audio. 
This was a request from the customer.

\item[Colour settings from Launcher use in PictoCreator]
Apply colour settings parsed from Launcher in PictoCreator.

\item[Update GUI Components]
Update GUI such that it uses gComponents, developed by another group in the GIRAF project.

\item[Save pictogram]
Save pictogram in database and its associated sound.

\item[Load pictogram from database]
Load a pictogram from the database into the canvas.

\item[Flatten Button]
Give flatten button the functionality to move selected entity to back of the canvas.

\item[Pictogram title in text on pictogram]
Customers have expressed interest in having the title of the pictogram written on the pictogram on either the top or the bottom.

\item[Tags for save dialogue]
Be able to add tags to ones pictogram when saving it and support this in the GUI.

\item[Rotation of Freehand Entities]
Freehand entities are rotating around their start point, but should rotate around the centre of the entity.

\item[Collision detection for high stroke width objects]
The current collision detection does not take the stroke width into account. 
This means that objects with a large stroke width are very hard to hit, if their center area is very small, even though one could swear they hit the area it covered.

\item[Freehand collision detection]
The collision detection should work so if you click the freehand drawing it should be marked.

\item[Freehand hitbox]
The hitbox have to follow the rotation, so the hitbox works as the line's hitbox.

\item[Preview Pictogram]
When trying to save a pictogram, the preview sometimes does not load.
\end{description}