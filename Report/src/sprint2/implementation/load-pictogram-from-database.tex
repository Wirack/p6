\subsection{Load Pictogram from Database}
The application should be able to load existing pictograms, so they can be modified to accommodate desired changes.
To do this we decided to use Pictosearch, which is an application that is also under development in the GIRAF project.
Pictosearch is a search tool that can find pictograms stored in the database.
The chosen pictogram is then sent to calling application, which in this case is Pictocreator.

The call to start Pictosearch also sends the guardian's ID, to allow access to their private pictograms, as well as the purpose of the call which is \textit{single} meaning Pictosearch can return at most one pictogram, as seen in lines $5-9$ in \lstref{lst:callPictosearch}.
If Pictosearch is not installed, a message will inform the user about it.
An event to receive the pictogram is create, called \textit{onActivityResult}, and checks whether it return a results or not.
If there is a result, another function is called which loads the pictogram bitmap from the database using the pictogram ID returned from Pictosearch, as seen in line 5 in  \lstref{lst:loadPictogram}.
The bitmap is then loaded into the canvas in the original scale.

\begin{lstlisting}[caption=Method used to launch Pictosearch., label=lst:callPictosearch]
private void callPictosearch(){
    Intent intent = new Intent();

    try{
        intent.setComponent(new ComponentName( "dk.aau.cs.giraf.pictosearch",  "dk.aau.cs.giraf.pictosearch.PictoAdminMain"));
        intent.putExtra("currentGuardianID", author);
        intent.putExtra("purpose", "single");

        startActivityForResult(intent, RESULT_FIRST_USER);
    } catch (Exception e){
        Toast.makeText( this, "Pictosearch er ikke installeret.", Toast.LENGTH_LONG).show();
    }
}
\end{lstlisting}

\begin{lstlisting}[caption=Method to load a pictogram from Id received from Pictosearch, label=lst:loadPictogram]
private void loadPictogram(Intent data){
    try{
        int pictogramID = data.getExtras().getIntArray("checkoutIds")[0];

        Bitmap pictogram = (PictoFactory.getPictogram(this, pictogramID).getImageData());

        drawFragment.drawView.loadFromBitmap(pictogram);
    }catch(Exception e){
        Log.e(TAG, e + ": No pictogram returned.");
    }
}
\end{lstlisting}