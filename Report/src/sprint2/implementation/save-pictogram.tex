\subsection{Save Pictogram}
A pictogram has some needed attributes that have to be saved together with the pictogram.
These attributes can be assigned in the save dialogue fragment and include the publicity of the pictogram, public or private, the tags following the pictogram, the inline text, and the name of the pictogram.
To save the pictogram, we first create a pictogram object and assign the needed attributes to it.
The sound is saved in the cache of the tablet in another dialogue, but now it is assigned to the pictogram object so it can be saved in the database.
Furthermore, the author's id is added as well as the bitmap from the canvas.
Once the attributes have been assigned, the pictogram is stored in the database via the provided methods from the database.
The database gives the pictogram an id and returns it back, so it can be used to create the relation between tags and pictogram.