\chapter{Role in the Multi Project}
%mål:
%fixe excisterende software
%integrere med andre projecter
%tilføj ny functionalitet så det passer til kunders behov
%all in all: gør det klar til release
Our role in the multi project environment is to improve on the foundations created by the previous developers.
These improvements can be fixing bugs in the existing code, integration with other projects in the multi project, or adding new functionality to match the customers request and expectation.
This goal is the reason that the focus is on developing working code which can be released to the customers, rather than optimal code that is not ready for the customers.


\section{Analysis of Organisational Context}
%scrum snak
%weekly meetings
%scrum of scrums
%sprintstart/sprintend
%custommers
In the multi project, there is around 60 people working together. 
To make this teamwork manageable, some organisation is needed.
Organising is done by utilising the SCRUM methodology, and since there are so many groups, a SCRUM of SCRUMS is used.
This means that there are daily SCRUM meetings, weekly meetings between the SCRUM teams, and meeting with the customers at the end of each sprint.


%eksperter/specialister/område ansvarlige

%tools: studio, jenkins, redmine, git

%colaboration
%informal communication between groups

