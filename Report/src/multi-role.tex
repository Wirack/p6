\chapter{Role in the Multi Project}
%mål:
%fixe excisterende software
%integrere med andre projecter
%tilføj ny functionalitet så det passer til kunders behov
%all in all: gør det klar til release
Our role in the multi project environment is to improve on the foundations created by the previous developers.
These improvements can be fixing bugs in the existing code, integration with other projects in the multi project, or adding new functionality to match the customers request and expectation.
This goal is the reason that the focus is on developing working code, which can be released to the customers, rather than optimal code that is not ready for the customers.

\section{Analysis of Organisational Context}
%scrum snak
%weekly meetings
%scrum of scrums
%sprintstart/sprintend
%custommers
In the multi project, there is around 60 people working together. 
To make this teamwork manageable, some organising is needed.
Organising is done by utilising the SCRUM methodology, and since there are so many groups, a SCRUM of SCRUMS is used.
This means that there are daily SCRUM meetings, weekly meetings between the SCRUM teams, and meeting with the customers at the end of each sprint.
The goal is to create an application that aids autistic people and their guardians in communication.
Each group gets a part of the application and further develops on it. 
The decisions involving parts of the application are put in the hands of the group responsible for said part, while decisions involving the overall application are discussed at the weekly meetings.

At the start of the development process, we had several meetings, with everyone present, where we did a transition with the previous development group to the current one.
During these early meetings we discussed how to organise the multi project and did decision based on our own knowledge and the previous development group's experience.

%tools: studio, jenkins, redmine, git
Tools for managing the development were also decided at these meetings.
GitHub was chosen as our configuration management tool, as this is easily available, a lot of developers have experience with it, and the previous group also worked with it, which made the transition easier.

Redmine was chosen as the main information sharing tool in regards to guides on how to use various parts, different standards such as the ones used for coding and meetings.
Redmine was also chosen because it had a great issue tracker, allowing developers to easily check what was being developed on the other parts and to check their progress.

Another tool we used was Jenkins, which automatically build the project and alerts certain people when the build fails.
Jenkins was also used to get the newest version of a part of the project, without the need to download the repository of the project and compile it.

The chosen IDE was Android Studio, which was suggested by the previous development group. Android Studio is an IDE designed specifically for Android devices making it a suitable choice.

%eksperter/specialister/område ansvarlige
Personnel from the development group were chosen as being in charge of the tools.
It was also the responsibility of these people to be knowledgeable of their tool.

As mentioned before, we had planned meetings with the customers at the end of each sprint, where all the parts of the application were presented with focus on the newly developed features.
At the start of the sprint we did another meeting with all the developers present, where decisions were made regarding what parts each development group was to develop in the coming sprint.
%colaboration
%informal communication between groups

