\subsection{Icon changes}
When changing the icons the focus was to make them easier to associate the functionality to the icons.
The idea behind the new choice of freehand drawing icon is what people associate with this functionality, and how it is represented in similar programs.
In such similar programs the standard representation of a freehand drawing is a pencil.
Due to this standard the icon in \figref{figure:old-freehand} was replaced by \figref{figure:new-freehand}.

\begin{figure}[h]
	\centering
	%---- linebreak	
	\begin{subfigure}[b]{0.45\textwidth}
		\centering
		\includegraphics[scale = 1]{media/freehandNew}
		\caption{New freehand drawing icon}
		\label{figure:new-freehand}
	\end{subfigure}
	\qquad
	\begin{subfigure}[b]{0.45\textwidth}
		\centering
		\includegraphics[scale = 0.1]{media/freehandOld}
		\caption{Old freehand drawing icon}
		\label{figure:old-freehand}
	\end{subfigure}
	\caption{Freehand drawing icons}
	\label{figure:freehand}
\end{figure}

The previous icon connected with this feature could give a misleading impression of the functionality that actually was connected to the feature.
The icon, seen in \figref{figure:old-select}, looks much like a tool that marks and area, and of such selects everything in this area, where it only selects one object.
To make the icon represent the functionality of the feature a new icon was deemed necessary.
The new icon was made to look like a hand, \figref{figure:new-select} as this gives a clear representation of only being able to select one object.

\begin{figure}[h]
	\begin{subfigure}[b]{0.45\textwidth}
		\centering
		\includegraphics[scale = 1]{media/selectNew}
		\caption{New selection icon}
		\label{figure:new-select}
	\end{subfigure}
	\qquad
	\begin{subfigure}[b]{0.45\textwidth}
		\centering
		\includegraphics[scale = 0.1]{media/selectOld}
		\caption{Old selection icon}
		\label{figure:old-select}
	\end{subfigure}		
	\caption{Selection icons}
	\label{figure:select}
\end{figure}