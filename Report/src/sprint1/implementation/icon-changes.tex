\subsection{Icon changes}\fxnote{En af de subsections vi kan fjerne hvis vi har for mange sider}
The previous development group working on this application \citep{misc:crocold} performed usability tests and through these, found that some of the icons were not intuitive to the users and thus needed changing.
When changing the icons, the focus is to make the icons easier to associate with the functionality of their buttons.
The idea behind the new choice of freehand drawing icon is what people associate with this functionality and how it is represented in similar applications.
In such similar applications the standard representation of a freehand drawing is a pencil.
Due to this standard the icon in \figref{figure:old-freehand} is replaced by \figref{figure:new-freehand}.

\begin{figure}[h]
	\centering
	%---- linebreak	
	\begin{subfigure}[b]{0.45\textwidth}
		\centering
		\includegraphics[scale = 0.1]{media/freehandOld}
		\caption{Old freehand drawing icon.}
		\label{figure:old-freehand}
	\end{subfigure}
	\qquad
	\begin{subfigure}[b]{0.45\textwidth}
		\centering
		\includegraphics[scale = 1]{media/freehandNew}
		\caption{New freehand drawing icon.}
		\label{figure:new-freehand}
	\end{subfigure}
	\caption{Freehand drawing icons}
	\label{figure:freehand}
\end{figure}

The previous icon for the selection tool could give a misleading impression of the functionality.
The icon, seen in \figref{figure:old-select}, looks much like a tool that marks an area, but the tool selects a single entity.
To make the icon represent the functionality of the feature a new icon is deemed necessary.
The new icon looks like a hand, see \figref{figure:new-select}, as this gives a clear representation of being able to select a single entity at a time.

\begin{figure}[h]
	\begin{subfigure}[b]{0.45\textwidth}
		\centering
		\includegraphics[scale = 0.1]{media/selectOld}
		\caption{Old selection icon}
		\label{figure:old-select}
	\end{subfigure}	
	\qquad
	\begin{subfigure}[b]{0.45\textwidth}
		\centering
		\includegraphics[scale = 1]{media/selectNew}
		\caption{New selection icon}
		\label{figure:new-select}
	\end{subfigure}
	\caption{Selection icons}
	\label{figure:select}
\end{figure}