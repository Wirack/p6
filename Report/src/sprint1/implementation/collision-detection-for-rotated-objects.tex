\subsection{Collision Detection for Rotated Objects}
The user should be able to select the object by clicking inside of the hitbox, even when the object is rotated.
This gave an issue when rotating an object the hitbox did not follow and to click on the object you had to click inside the misplaced hitbox.

The reason for this was that when the object was created the hitbox was set but never changed after rotating the object.
Therefore, a change was made to the hitbox property such that the top left corner of the hitbox was set to the minimum x and y value of an objects coordinates.


\begin{lstlisting}
protected void changeHitbox(){
	FloatPoint one = rotationMatrix( -(getWidth()/2), -(getHeight()/2));
	FloatPoint two = rotationMatrix((getWidth()/2), -(getHeight()/2));
	FloatPoint three = rotationMatrix((getWidth()/2), (getHeight()/2));
	FloatPoint four = rotationMatrix( -(getWidth()/2), (getHeight()/2));
	
	hitboxTopLeft = new FloatPoint(findMin(one.x, two.x, three.x, four.x),
	                               findMin(one.y, two.y, three.y, four.y));
	hitboxWidth = (getCenter().x - hitboxTopLeft.x)*2;
	hitboxHeigth = (getCenter().y - hitboxTopLeft.y)*2;
}
\end{lstlisting}



%movehitboxwithline
%collisiondetectionforline
%collisiondetectionforrotatedrectangles
%rotateline
%collisiondetectionforrotatedellipses