\section{Croc}
%hvad croc er
\textit{Croc} is a part of the \textit{GIRAF} software package. 
It is meant to support manual creation of pictograms.
In that regard, to create a pictogram, the user can take pictures, record audio, and draw their own pictures.
These options can then be used in any combination as the user sees fit for a given pictogram.

%hvem bruger det
The target audience are guardians and parents, contrary to other parts of \textit{GIRAF}. 
For that reason, the design focus is on usability for guardians as well as to make suitable pictograms, rather than educational or entertainment purposes.

%Crocs kontekst - database
\textit{Croc} is developed as an application that may be launched from other applications in the \textit{GIRAF} project.
When a pictogram has been created in \textit{Croc} it is meant to be stored in a database to make it accessible from other applications.

%meta til subsections
In order to achieve these goals of \textit{Croc}, features has been implemented although not without issues.
\subsection{Features}