\section{Croc}
%hvad croc er
Croc is a subprogram in the GIRAF development project. 
It is meant to support manual creation of pictograms.
In that regard, to create a pictogram, the user can take pictures, record audio, and draw their own pictures.
These options can then be used in any combination as the user sees fit for a given pictogram.

%hvem bruger det
The target audience are guardians and parents, contrary to other subprograms of GIRAF. 
For that reason, the design focus is on usability for guardians as well as to make suitable pictograms, rather than educational or entertainment purposes.

%Crocs kontekst - database
Croc is developed as a program that may be launched from other subprograms in the GIRAF project.
When a pictogram has been created in Croc it is meant to be stored in a database to make it accessible from other subprograms.

%meta til subsections
In order to achieve these goals of Croc, features has been implemented although not without issues.
\subsection{Features}
\subsection{Issues}
As mentioned before, in this sprint we will add missing features and resolve bugs in the program we have been assigned to.
Some of these issues are found by the group working on the program last year, and the rest have been found by testing the functionality of the program.
The different issues found are prioritised and estimated through poker planning.
They are described later in this chapter, when the certain issue is discussed.

The issues listed in \tabref{table:oldissues} are found in the report from last year \citep{misc:crocold}
For some of the listed issues, ideas of how to resolve the issues have been proposed.

\begin{table}[h]
\begin{tabular}{|p{5cm}|c|c|}
\hline 
Issues & Priority & Story points \\ 
\hline 
Collision detection for rotated objects & High & 13 \\ 
\hline 
Change background colour & Low & 8 \\ 
\hline 
Auto-focus camera & Low & 5 \\ 
\hline 
Microphone noise & Medium & 8 \\ 
\hline 
Store and load from database & Very High & 5 \\ 
\hline 
Customised object layers order & low & 8 \\ 
\hline 
Eraser & High & 13 \\ 
\hline 
Push object to the back & low & 3 \\ 
\hline 
Improve resize interaction & Medium & 5\\ 
\hline 
Update GUI pictures & Medium & 2 \\ 
\hline 
Implement back button & Medium & 2 \\ 
\hline 
\end{tabular}
\caption{Issues found by the previous development group.}
\label{table:oldissues}
\end{table}

After testing the functionality of the current program, additional issues were discovered that were deemed to be necessary to resolve.
These issues are listed in \tabref{table:newissues}.

\begin{table}[h]
\begin{tabular}{|p{5cm}|c|c|}
\hline 
Issue & Priority & Story Point \\ 
\hline 
Change stroke width & Medium & 2 \\ 
\hline 
Loading pictures from camera in optimal size & High & 2 \\ 
\hline 
Colour swap & High & 5 \\ 
\hline 
Straight line colour & High & 5 \\ 
\hline 
Clear button & Medium & 3 \\ 
\hline 
Save canvas state & Medium & 5 \\ 
\hline 
Microphone dialogue box & Low & 5 \\ 
\hline 
Scale camera pictures & High & 5 \\ 
\hline 
Preview recording & High & 5 \\ 
\hline 
\end{tabular}
\caption{Issues found after feature testing the program.}
\label{table:newissues}
\end{table}

A suitable amount of issues have been selected to ensure that these issues are resolved before the sprint ends.
As sprints have a fixed duration, some of the listed issues have to be discarded.
Some issues are easily discarded, e.g. the database issues, since they depend on another group that are currently not developing the related functionality that the issue requires.
Another reason to discard an issue is because of its priority and story point assigned to it.
After a group discussion a final list is determined and can be seen in the following list.

\begin{itemize}
	\item Change stroke width
	\item Loading pictures from camera in optimal size
	\item Colour swap
	\item Straight line colour
	\item Update GUI pictures
	\item Clear button
	\item Collision detection for rotated objects
	\item Save canvas state
	\item Microphone dialogue box
	\item Scale camera pictures
	\item Preview recording
\end{itemize}