\section{Croc}
%hvad croc er
\textit{Croc} is a part of the \textit{GIRAF} software package. 
It is meant to support manual creation of pictograms.
In that regard, to create a pictogram, the user can take pictures, record audio, and draw their own pictures.
These options can then be used in any combination as the user sees fit for a given pictogram.

%hvem bruger det
The target audience are guardians and parents, contrary to other parts of \textit{GIRAF}. 
For that reason, the design focus is on usability for guardians as well as to make suitable pictograms, rather than educational or entertainment purposes.

%Crocs kontekst - database
\textit{Croc} is developed as an application that may be launched from other applications in the \textit{GIRAF} project.
When a pictogram has been created in \textit{Croc} it is meant to be stored in a database to make it accessible from other applications.

%meta til subsections
In order to achieve these goals of \textit{Croc}, features has been implemented although not without issues.
\subsection{Features}
A lot of features of \textit{Croc} are tied to the user interface, and for that reason it makes sense to present the \textit{Croc} UI.
%canvas features
In order to get an idea of how the \textit{Croc} UI looks like, see \figref{fig:croc-old-canvas}. 

The UI is divided into two parts, the upper serving as a menu-bar allowing access to the camera, audio recorder, help view, closing the application, or saving the pictogram.
The lower part changes depending on the mode chosen.
In the figure, the lower part shows the UI for the following features.

\begin{itemize}
     \item Selection tool
     \item Freehand, rectangle, line, and ellipsis drawing tool
     \item Choosing a colour
     \item Importing camera picture
     \item Preview of chosen colours
     \item Changing stroke width
\end{itemize}

Other features exists for the camera and audio recording, which include switching between black/white and colour pictures, take picture, and start/end recording.
For an illustration of the camera and audio recorder, see \appref{app:crocUI}.

\begin{figure}[h]
     \centering
     \includegraphics[width=\textwidth]{CrocOldCanvas}
     \caption{Screenshot of canvas UI.}
     \label{fig:croc-old-canvas}
\end{figure}