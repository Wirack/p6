\section{PictoMediaPlayer}\label{sec:pictomediaplayer}
Associating sound to pictograms was stated as of great importance from the customers.
For that reason a library is developed that provides an easy to use sound playing service.

Both \textit{Piktotegner} and \textit{Piktooplæser} need the sound playing feature. 
\textit{Piktotegner} needs it to get a preview of the sound recorded, to be saved when creating a pictogram.
\textit{Piktooplæser} needs the feature as the key feature of \textit{Piktooplæser} is to get the sound of the pictograms played sequentially. 

When initialising the library it needs a context and can additionally take a path for the sound file, a pictogram, or a bytearray. 
The context is needed for the \textit{PictoMediaPlayer} to know where to play the sound.
In \lstref{lst:constructor} we see a constructor for the \textit{PictoMediaPlayer}.

\begin{lstlisting}[caption={Constructor for \textit{PictoMediaPlayer}.},label={lst:constructor}]
public PictoMediaPlayer (Context activity, String path)
{
    this.activity = activity;
    assignMediaPlayer();
    setDataSource(path);
}
\end{lstlisting}

When assigning a datasource to the \textit{PictoMediaPlayer} the \textit{setDataSource} method, seen in \lstref{lst:setdatasource}, is called. The method releases an old \textit{MediaPlayer} and assigns a new one if sound has already been attached to the \textit{MediaPlayer}.
Then the \textit{MediaPlayer} gets a file descriptor assigned as source.

\begin{lstlisting}[caption={SetDataSource method of \textit{PictoMediaPlayer}.},label={lst:setdatasource}]
public void setDataSource(String path)
{
    if(hasSound)
    {
        mediaPlayer.release();
        assignMediaPlayer();
    }

    try
    {
        FileInputStream fileInputStream = new FileInputStream(path);

        mediaPlayer.setDataSource(fileInputStream.getFD());
        hasSound = true;
    }
    catch (IOException e)
    {
        e.getStackTrace();
    }
}
\end{lstlisting}

The \textit{OnCompletionListener}, seen in \lstref{lst:completelistener}, is necessary for external sources to detect when a sound is finished playing. The listener allows external sources to respond to a sound finished playing, e.g. a button icon change.

\begin{lstlisting}[caption={onCompletionListener method of \textit{PictoMediaPlayer}.}, label={lst:completelistener}]
private final MediaPlayer.OnCompletionListener onCompletionListener = new MediaPlayer.OnCompletionListener()
{
    @Override
    public void onCompletion(MediaPlayer mp) 
    {
        isPlaying = false;
        if(customListener != null)
        {
            customListener.soundDonePlaying();
        }
    }
};

public interface CompleteListener
{
    public void soundDonePlaying();
}
\end{lstlisting}

For \textit{Piktooplæser} to play the sound of multiple pictograms in succession we make the \textit{playListOfPictograms} method as seen in \lstref{lst:playlist}. 
We first store the old listener in a temporary variable and stop playing sounds. Then we overwrite the method which is run after a sound is complete, so that it plays the sound of the next pictogram in the list. When all pictograms in the list have been read the old listener is restored.

\begin{lstlisting}[caption={playListOfPictograms method of \textit{PictoMediaPlayer}.}, label={lst:playlist}]
public void playListOfPictograms(ArrayList<Pictogram> pictogramList)
{
    pictogramListIndex = 0;
    this.pictogramList = pictogramList;
    TempCompleteListener = customListener;

    if(isPlaying)
    {
        stopSound();
    }
    this.setCustomListener(this);
    this.setDataSource(pictogramList.get(pictogramListIndex));
    this.playSound();
}

@Override
public void soundDonePlaying()
{
    pictogramListIndex ++;
    if (pictogramListIndex < pictogramList.size() && pictogramList.get(pictogramListIndex) != null)
    {
        setDataSource(pictogramList.get(pictogramListIndex));
        playSound();
    }
    else
    {
        this.setCustomListener(TempCompleteListener);
        pictogramListIndex = 0;
    }
}
\end{lstlisting}

To illustrate the ease of use of the \textit{PictoMediaPlayer} we created the example seen in \lstref{lst:mediaplayerexample}. 
We here see how a \textit{PictoMediaPlayer} is used to play a sound and writing ``Sound is finished playing'' when the sound is finished playing.

\begin{lstlisting}[caption={Example of \textit{PictoMediaPlayer}.},label={lst:mediaplayerexample}]
public class example implements CompleteListener
{
    private void examplemethod(Context context, String path)
    {
        PictoMediaPlayer pictoMediaPlayer = new PictoMediaPlayer(context, path);

	pictoMediaPlayer.setCustomListener(this);
        pictoMediaPlayer.playSound();
    }

    @Override
    public void soundDonePlaying()
    {
        System.out.print("Sound is finished playing");
    }
}
\end{lstlisting}

%Disposition
%interest in the feature from both groups

% why?
% easy use
% just specify a filepath and all features are there. 



% List of features:
    % abstracts away of the states of android's MediaPlayer
    % Play and stop music
    % supports customer listerner for listening to when sound is done playing.
    % 
