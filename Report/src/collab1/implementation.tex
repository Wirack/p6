\subsection{Implementation}
The implementation of the TTS functionality is created by opening an url connection to \textit{Google Translate}'s TTS method.
When the url connection is established, a stream starts to return bytes.
These bytes are then saved in an array so they can be translated into sound afterwards.
How to get the bytes and save them can be seen in \lstref{lst:download-sound}.

\begin{lstlisting}[caption={Download Sound event},label={lst:download-sound}]
private void DownloadFile() {
    try{
       URL url = new URL(soundURL);

       URLConnection urlConnection = url.openConnection();
       InputStream inputStream = urlConnection.getInputStream();
       BufferedInputStream bis = new BufferedInputStream(inputStream);
       ByteArrayBuffer byteArrayBuffer = new ByteArrayBuffer(50);
       int current = 0;
       while ((current = bis.read()) != -1)
           byteArrayBuffer.append((byte) current);

       SoundData = byteArrayBuffer.toByteArray();
   }[...]
}
\end{lstlisting} 

Before calling this method, it must be ensured that the device has a valid internet connection, because if there is no internet connection you are not able to connect to \textit{Google Translate}.
To check if there is an internet connection, a connection manager checks if it can get network activity.
If the connection manager does not find an internet connection the \textit{DownloadFile} method cannot be called.