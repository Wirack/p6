\subsection{Prototypes}
Before the user interface (UI) changes are performed in the application, paper prototypes are created to get a first intuition of how the UI should look.
These prototypes are based upon the talk with the customers \citep{misc:drazenko, misc:pernille}.
 
\subsubsection*{Rearrange UI Components}
Based on the request from the customers, the UI components of the drawing surface should be rearranged.
Two prototypes were drawn for this, which can be seen in \figref{fig:v1drawfrag} and \figref{fig:v2drawfrag}.
The idea of \figref{fig:v1drawfrag} was to move the components that had to do with the creation of a pictogram down in the \textit{drawFragment}. Furthermore, to arrange the options for the drawing tools near the tools themselves.
However, it was deemed that the program became asymmetrical due to this change.

\begin{figure}[h]
     \centering
     \includegraphics[angle=270, scale=0.5, trim = 1.5cm 3cm 5.5cm 5.5cm,clip]{sprint3/v1-drawfragment.pdf}
     \caption{Version 1 drawfragment.}
     \label{fig:v1drawfrag}
\end{figure}

The idea of \figref{fig:v2drawfrag} was to keep everything drawing related in the \textit{drawFragment} and additional features in the top bar. In addition, the options of the drawing tools were separated from the tools themselves, to regain symmetry.

\begin{figure}[h]
     \centering
     \includegraphics[angle=90, scale=0.5, trim = 4cm 5cm 2.5cm 4cm,clip]{sprint3/v2-drawfragment.pdf}
     \caption{Version 2 drawfragment.}
     \label{fig:v2drawfrag}
\end{figure}

Parts of the two prototypes were found satisfactory, and as of such a combination of the UI was performed.
It resulted in still having the separation of tools and their options, but making the \textit{drawFragment} being a more general layout, featuring the creation of pictograms instead of the drawing part solely.

\subsubsection*{Camera Dialogue}
The customers found it unintuitive to be able to take multiple pictures at a time and then later load some of the pictures for a pictogram.
For that reason, the camera dialogue should be changed to only be able to take one picture at a time.
When the picture is taken it can be verified, and then added to the pictogram, or retry with a new picture.

\begin{figure}[h]
     \centering
     \begin{subfigure}{0.45\textwidth}
          \includegraphics[angle=90, width = \textwidth, trim = 3cm 3cm 2.5cm 10cm,clip]{sprint3/camera-takepicture.pdf}
          \caption{Take Picture}
          \label{fig:cam-takepic}
     \end{subfigure}      
     \begin{subfigure}{0.45\textwidth}
          \includegraphics[angle=270, width = \textwidth, trim = 1cm 7cm 4.5cm 6.5cm,clip]{sprint3/camera-verify.pdf}
          \caption{Verify Picture}
          \label{fig:cam-verifypic}
     \end{subfigure}
     \caption{Camera Dialogue}
     \label{fig:cam-dialogue}
\end{figure}

\figref{fig:cam-dialogue} shows a prototype of the camera dialogue, where inspiration has come from popular image applications such as the inbuilt iOS camera and Instagram.\fxwarning{reference her.}
Seen in \figref{fig:cam-takepic} is the dialogue as it should look like when preparing to take a picture of a tree.
The button to the lower left is to swap between colour and black and white mode, which is a request of the customers.
In the lower right corner, a button to swap between cameras can be seen.
Finally in the lower center, a button to take the picture can be seen.

When the picture is taken, you are transferred to the UI seen in \figref{fig:cam-verifypic}, where you have the option to accept the taken picture or take a new picture.

In the whole camera dialogue you always have the option to quit the dialogue by pressing the button in the upper right corner.