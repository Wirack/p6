\subsection{DrawStack}
After a pictogram is created and saved with \textit{Piktotegner}, the customers want to be able to load it again, without loss of pictogram information.
This information consists of the \textit{drawStack} of a pictogram, such that different painting entities can be edited individually and not a singular bitmap.
If the pictogram is saved in the database from some other source that is not \textit{Piktotegner}, it is the singular bitmap you load.
The \textit{drawStack} issue consists of saving and loading the \textit{drawStack}, and is described hereafter.

\subsubsection*{Saving the DrawStack}
The idea to save the \textit{drawStack} on the database is to convert the \textit{drawStack} into a byte array.
Java provides this service if you implement the \textit{Serializable} interface, which is a keyword where you do not need any implementation if the data types of the fields are all \textit{Serializable}.
However, \textit{RectF} and \textit{Paint} from the Android Graphics Library do not implement the \textit{Serializable} interface.

At first this is tried to be solved by making \textit{Serializable} versions of those classes, which then extends \textit{RectF} and \textit{Paint} respectively but implement the \textit{Serializable} interface.
This method resulted in the \textit{drawStack} being able to be converted to a byte array, however, the information of the stroke width and colour of entities is lost.
This is due to the \textit{Paint} and \textit{RectF} classes again including fields that are not \textit{Serializable}.
As of such, integers are introduced to contain the information of those objects, in order to bypass the usage of those objects as fields and thereby be able to convert the \textit{drawStack}tack to a byte array.
\textit{Paint} and \textit{RectF} can be solved in this way, but when containing a camera picture represented as a bitmap in the \textit{drawStack}, difficulties arise when converting to a byte array.

If the \textit{drawStack} contains a bitmap, a solution is not as easy as introducing integers to hold information.
A library is instead tried to manually convert the bitmap to a byte array and back again.
However, for the current sprint, a satisfying solution is not found, because if the bitmap has to retain a certain quality, the conversion is too resource demanding.
As of such, for future work, a way to save the \textit{drawStack}, when a bitmap is an element thereof, is needed.
At present, the \textit{drawStack} gets saved if there is no bitmap for the pictogram, else the singular bitmap saving and loading is performed.

\subsubsection*{Loading the DrawStack}
For loading a \textit{drawStack}, it is not a problem, as the \textit{drawStack} has already been saved as a byte array and the classes have been made \textit{Serializable}.
Then it is a matter of using the developed method to convert the byte array to a \textit{drawStack}, see \lstref{lst:desrialize}.

\begin{lstlisting}[caption={Method for converting byte array to object}, label=lst:desrialize]
public static Object deserialize(byte[] data) throws IOException, ClassNotFoundException {
	ByteArrayInputStream in = new ByteArrayInputStream(data);
	ObjectInputStream is = new ObjectInputStream(in);
	return is.readObject();
}
\end{lstlisting}
