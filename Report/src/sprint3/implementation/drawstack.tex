\subsection{Drawstack}
After a pictogram is created and saved with \textit{Pikto Tegner}, the customers want to be able to load it again, without loss of pictogram information.
This information consists of the drawstack of a pictogram, such that different painting entities can be edited individually and not a singular bitmap.
If the pictogram is saved in the database from some other source that is not \textit{Pikto Tegner}, it is the singular bitmap you load.

\subsubsection*{Save Drawstack}
The idea to save the drawstack on the database was to convert the drawstack into a byte array.
Java can provide this service if you implement the \textit{Serializable} interface, which is a keyword where you do not need any implementation if the data types of the fields are all \textit{Serializable}.
However, \textit{RectF} and \textit{Paint} from the Android Graphics Library did not implement the \textit{Serializable} interface.

At first this was tried to be solved by making \textit{Serializable} versions of those classes, which then extended \textit{RectF} and \textit{Paint} respectively but implement the \textit{Serializable} interface.
This method resulted in the drawstack being able to be converted to a byte array, however, the information of the stroke width and color of entities were lost.
This was due to the \textit{Paint} and \textit{RectF} classes again including fields that were not \textit{Serializable}.
As of such, integers were introduced to contain the information of those objects, in order to bypass the usage of those objects as fields and thereby be able to convert the drawstack to a byte array.
\textit{Paint} and \textit{RectF} could be solved in this way, but when containing a camera picture represented as a bitmap in the drawstack, difficulty arose when converting to a byte array.

If the drawstack contains a bitmap, a solution is not as easy as introducing integers to hold information.
A library is instead tried to manually convert the bitmap to a byte array and back again.
However, for the current sprint a satisfying solution is not found, because if the bitmap had to retain a certain quality, the conversion was too resource demanding.
As of such, for future work, a way to save the drawstack when a bitmap is an element thereof is needed.
At present, the drawstack only gets saved if there is no bitmap for the pictogram, else the singular bitmap saving and loading is performed.

\subsection*{Load Drawstack}
For loading a drawstack, it is not a problem, as the drawstack has already been saved as a byte array and the classes have been made \textit{Serializable}.
Then it is just a matter of using the developed method to convert the byte array to a drawstack, see \lstref{lst:desrialize}.

\begin{lstlisting}[caption={Method for converting byte array to object}, label=lst:desrialize]
public static Object deserialize(byte[] data) throws IOException, ClassNotFoundException {
	ByteArrayInputStream in = new ByteArrayInputStream(data);
	ObjectInputStream is = new ObjectInputStream(in);
	return is.readObject();
}
\end{lstlisting}
%save drawstack
     %Serializable
     %Android classes that is not serializable
          %RectF
          %Paint
               %usage of integers          
          
          %meta to bitmap

     %Problem with bitmaps <- too resource consuming

%load drawstack
     %convert byte[] to list and invalidate