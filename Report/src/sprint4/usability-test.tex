\section{Usability Test}\label{sec:usability-test}
To ensure that the customers are able to use the application and that it is easy to understand, a usability test is performed.
The application should be easy to understand since the customers are not used to Android tablets.
Moreover, the customers are receiving a tablet with the application installed and are therefore not told how to use the application beforehand.

To make a usability test, several tasks are constructed which makes the testees try all parts of the application. 
The testees then have to try to complete these tasks without help.
If they are unable to complete the tasks without help, they can ask the tester for help.
Problems are then described and put into categories depending on how the testees complete the tasks.

The problems are categorised into three categories, which are cosmetic, serious, and critical.
Problems that are categorised as cosmetic are defined as tasks that are slowing the testees down by a few seconds.
When a problem is categorised as serious, the testees are unable to solve the problem without any help or are stuck for several seconds.
The worst type of problem is critical, this problem occurs when the testees are unable to solve the problem and the person sitting next to the testees has to show them how to perform the task.

The usability test is performed in the end of this sprint and the issues found are therefore not resolved in this sprint.
The tests are performed with the help of two testees, who are also customers.

\begin{table}[h]
	\centering 
	\rowcolors{2}{gray!25}{white}
	\begin{tabular}{|c|c|c|}
		\rowcolor{gray!50}
		\hline 
		Problems & Description & Category \\ 
		\hline
		P1 & Change camera picture to black and white & Cosmetic \\ 
 
		P2 & Clear canvas & Cosmetic \\ 
 
		P3 & Drawing entities & Serious \\ 

		P4 & Move entities behind another entity & Critical \\ 
 
		P5 & Swap colours & Critical \\ 
 
		P6 & Add tag & Cosmetic \\ 
 
		P7 & Delete entity & Serious \\ 
 
		P8 & Colour picker & Cosmetic \\ 
 
		P9 & Play sound & Cosmetic \\ 
		\hline 
	\end{tabular} 
	\caption{Discovered usability problems.}
	\label{tab:usability-problems}
\end{table}

\subsection*{P1 - Change Camera Picture to Black and White}
A problem occurred when the testees had to take a picture of their chair in black and white.
The problem was that one of the teestees took the picture and thought that they could change the colour of the picture afterwards.
The other testee did not read the tasks well enough and therefore forgot to pick black and white.

The reason why this is a cosmetic problem is that the first testee had some problems finding out how to change the colour of the picture.
However, the second testee found the colour changer right when the testee was told that it should be in black and white.

\subsection*{P2 - Clear Canvas}
When the testees had to clear the canvas, after drawing, they had problems finding the button to do so.
However, after looking for a few seconds the testees were able to locate the button, which is why the problem is cosmetic.

\subsection*{P3 - Drawing Entities}
Drawing entities is done by swiping the screen.
The start position of the swipe is then the first corner and the end of the swipe is the opposite corner, the entity is then drawn using these corners.
What the testees did when trying to draw an entity was that they pressed the screen and believed that the entity would be created.

This problem is a serious problem since the testees used several seconds figuring how to draw an entity.
However, none of the testees needed help to be able to draw an entity, and for that reason it is not deemed a critical problem.

\subsection*{P4 - Move Entities Behind Another Entity}
Moving an entity behind another entity was a problem for one of the testees, since the testee was unable to understand where the button to do this was.
When told which button pushed an entity back, the testee felt that the icon was not understandable.
Furthermore, the testee thought that the button would push an entity in front of another instead of behind an entity.
Therefore, the testee needed help figuring out how to use this functionality.
Since the testee had so many problems with this task, it is deemed a critical problem.

\subsection*{P5 - Swap Colours}
The button to swap colours was not understandable by the testees.
Furthermore, none of the testees figured out how the button worked even after getting told how it works.
For that reason this problem is a critical problem and should be fixed in the next sprint of \textit{Piktotegner}.

\subsection*{P6 - Add Tag}
The problem occurred when one of the testees should add a tag to the tag list.
The testee did, however, only need a little help to figure out how to add the tag to the list.
As just one of the testees were not able to add it to the list and had few problems doing this, the problem is deemed to be a cosmetic problem.

\subsection*{P7 - Delete Entity}
When deleting an entity, the user have to select the selection tool and then select the entity that has to be removed.
However, none of the testees understood that they had to select the selection tool to do this.
Both of the testees thought that there would be an eraser to remove entities.
Furthermore, both of the testees tried to double press the entity because they thought that would select the entity.
However, after few seconds and a bit of help they were both able to see that selecting an entity would allow them to delete it.
Therefore the problem is deemed to be a serious problem. 

\subsection*{P8 - Colour Picker}
One of the testees had a problem with the colour picker.
The problem was that the testee thought that pressing the button that picks the last picked colour would open the colour picker.
However, after several seconds the testee figured out that pressing the colour picker would lead them to picking the colour.
Since only one of the testees had problems, it is deemed to be a cosmetic problem.

\subsection*{P9 - Play Sound}
To play the sound of a pictogram, the user has to go into the recording dialogue, which happens by pressing the record button.
The testees did not find this understandable as the record button should be for recording sound only.
However, the testees, after a few seconds, found that the record dialogue would be able to play the sound of a pictogram, which is why the problem is a cosmetic problem.
%Suggested solutions