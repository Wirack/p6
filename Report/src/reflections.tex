\chapter{Reflections}
Working in a multi project setting has been a great learning experience.
Starting the multi project, confusion existed as to how the project should be structured, as the project guidelines have been very different from previous semesters.

That being said, after the first couple of weeks getting used to the setting, the development was fun.
However, it requires that you are willing to cooperate with other groups, and is necessary to get the whole software package to be consistent.
Furthermore, it helped loosen up the project groups and provided a more social environment than the previous semesters.
When questions arose, it was a matter of knocking on the door of the other group's room, and the issue would be quickly resolved.

We found SCRUM a good technique, as when groups had problems, they could be found before too many resources were spent.
The product and sprint backlogs of most of the applications were organised in the tool called \textit{Redmine}, and we found this a useful tool.
It was useful because it made you able to monitor your own project as well as the whole multi project, to see if you are on track.
In addition to this, it was a way to give a concise description of each issue and ensure that no issues were forgotten.

In order to estimate the issues, to ensure that the sprint backlogs were fitting, we used planning poker.
We found planning poker a useful tool, and is highly recommended for future groups working with an agile methodology as it is a low resource estimation technique, which at the same time, for us, was fairly accurate.

\textit{Git} was also a useful tool, as it helped ensure version control, and is recommended for future students.
In addition to this, because we work in a big multi project, it is mandatory to have a tool such as \textit{Git}.
The reason for this is that we depend on libraries developed by other applications, such as the database library and pictogram library.
You do not have to adapt to changes in their developed libraries immediately, where you could risk getting interrupted all the time, to adapt to the changes made in those libraries. 
Instead you can wait to update your submodules for a fitting time in your schedule.
However, it takes some time getting used to \textit{Git} as submodules were good but some \textit{Git} structure was problematic.
This occurred if we were dependant on a submodule with which we had a shared submodule.
The submodule would then use our submodule which may have another revision and would have to be adjusted thereto.
For future students, it is recommended to use a development branch if you are making some unstable changes to your code.
Furthermore, before pushing you should ensure that your code is compilable.

In relation to this comes the automatic build service \textit{Jenkins}. 
It was set to automatically build the different changes when new revisions were added to \textit{Git}. 
People who may have pushed non-compilable code would then be notified, such that they could correct their mistakes.
As good a tool as \textit{Jenkins} was, it got in use late in the development cycle, and if we were to do the multi project development again, we would recommend getting \textit{Jenkins} in use as early as possible.

In order to also ensure that we had a role in the multi project, we had a person responsible for making icons for all the applications in the multi setting.
This ensured that we also provided a role other than developing the \textit{Piktotegner} application, in order to feel that we provided more to the whole project. 
Future students are also recommended to engage in such activities, being it icon creation, \textit{Git}, \textit{Jenkins}, or \textit{Redmine} responsibility.

Customer communication was a positive experience, although it was difficult to establish contact.
This is found to primarily be due to not contacting the customers in timely manner.
As the other responsibility roles, we also had a person responsible for establishing contact with the customers.
We found that usability tests was performed too late, as it was done in the two final days of sprint 4, thus the experiences from the usability tests could not be implemented in the application.
In the future, it would be wise to perform the usability tests earlier, in order to correct on the issues found during the tests.

During development of the application, the features developed were based on customer communication as the customers wanted a stable application, so those improvements were made autonomously.
There was, however, one feature requested by some of the customers that did not get implemented, the eraser tool.
The eraser tool was attempted to be developed, but as the drawing is based on vector graphics, deleting parts of an entity did not make much sense. 
When introducing a customer to the ability to removal of one entity at a time as an alternative, it was agreed upon to be a suitable replacement.

To summarize, the multi project was a great learning experience and is recommended to try for any student, as you get to work in a very different setting than what you are used to in previous semesters. 
Furthermore, we found that the development process went well, and the contact with the customers was appreciated as it is for them the application is developed for.


% Experience gain from working in multi-project
% cooperation
% large software package
% code dependencies and requests
% arbejde med SCRUM
% icon shizzle
% knock knock who's there
% customer communication
% 

%reflections and knowledge to be passed on to future students
