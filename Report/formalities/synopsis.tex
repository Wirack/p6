We develop a pictogram application, \textit{Piktotegner}, which is a part of a larger software package called \textit{GIRAF}.
\textit{GIRAF} is an autism communication software package developed for android devices.
The pictograms made in \textit{Piktotegner} can be saved and used in the other applications of \textit{GIRAF}.

We determine how to rotate and resize drawn objects as well as the thought process of designing an intuitive user interface.

Finally, a description of collaboration with other teams of the \textit{GIRAF} project is given, as well as the outcome and the experience gained thereof.
This description includes working in a large system setting and collaborating with other teams to develop a shared library.