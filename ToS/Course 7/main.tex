\chapter*{Exercise 2}
\section*{Purpose and Functionality of System}
The purpose of the system is to create,modify, copy and reuse pictograms. The software provides different painting tools, which is freehand, circle, square, and line drawing, in addition to rotation and resizing of those elements. 
Relevant criteria for measuring of its performance is time spent on pictogram creation, modification, copying and reusing.

+painting for autistic people
\section*{Alternative Existing Solution}
An existing alternative solution to compare the software with can be one of the following:
\begin{itemize}
     \item Drawing manually with pen and paper
     \item Drawing with other painting tools such as MSPaint, Photoshop, InkScape
\end{itemize}

\section*{Setup of experiment to collect empirical data for comparison}
Can do three way comparison between the general purpose painting tool, our tool and drawing by hand.

Can setup experiment with measurement of
time with chronometer.

usability test for: (should not be us who have developed one of the software solutions)
usability <-- can be hard to measure

questionaires for:
user satisfaction (scale from 1 to 10)

\section*{null hypothesis}
Wilcoxon as results will be tied to each person, aka. test cases. So a set of measurements for the same test case.

Risk is if the distribution for the different software tools on the test cases are not somewhat similar.

Also, problem with people not liking the system but really fast, and other way around. 
And for people who like the software, but are slow as they need practice to use it.


null hypothesis is:

our solution takes longer time to create a pictogram than using a general purpose painting solution.

Our solution satisfies the user less than solution B


--- we need numerical measurements.
