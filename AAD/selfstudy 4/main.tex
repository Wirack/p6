
\documentclass[12pt,a4paper]{report}
\usepackage[utf8]{inputenc}
\usepackage{amsmath}
\usepackage{amsfonts}
\usepackage{amssymb}
\usepackage{graphicx}
\usepackage{listings}
\usepackage{fancyhdr}
\usepackage{parskip}
\usepackage{rotating}
\pagestyle{fancy}
\chead{2.1.10 - sw608f14 - Daniel S. F., Lars A, Mathias W. P. \& Søren S. A.}

\lstset{mathescape = true}
\usepackage{amsthm}
\begin{document}
\section*{Selfstudy 4}
\section*{17-2}
Was answered in last self-study.

\section*{33-1}
\subsection*{a.}
We can use Jarvis's march until Q is empty.  Its cost is $O(n*h)$, so if we run it several times for j layers such that all points have been wrapped in a layer the cost will be $O(n*h_1 + n*h_2 .... n*h_j$. And we have $\sum\limits_{i=0}^j{h_i} = n$. So the worst case running time will be $O(n^2)$

\subsection*{b.}
Proof by contradiction.
First we see how we could use a convex layers algorithm to sort real numbers.
We could make the n real numbers into points where their x coordinate is their real number value and the y coordinate is 0.
This would result in the real numbers being chained and could be outputted in the correct order in $n$ time after being chained.
so if we had an algorithm for the convex layers problem that could run faster than $n*lg(n)$ we would  be able to sort faster than $n*lg(n)$ which is impossible.

\section*{33-2}
\subsection*{a.}
We observe that $y_i$ is the maximal $y$ value for the $i$'th layer, as if it was not the maximal $y$ value, it would be dominated by the other points in the layer, as they already have a higher x value.
We then take a look at $y_i$ and $y_{i+1}$ we already know that $y_i \geq y_{i+1}$ but it cannot be equal, because then the point the $y_{i+1}$ value is connected to would be in the $i$'th layer. So we have that $y_{i} > y_{i+1}$.
And thus $y_1 > y_2 > ... > y_k$

\subsection*{b.}


\end{document}