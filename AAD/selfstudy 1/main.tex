\documentclass[12pt,a4paper]{report}
\usepackage[utf8]{inputenc}
\usepackage{amsmath}
\usepackage{amsfonts}
\usepackage{amssymb}
\usepackage{graphicx}
\usepackage{listings}
\usepackage{fancyhdr}
\pagestyle{fancy}
\chead{2.1.10 - sw608f14 - Daniel S. F., Lars A, Mathias W. P. \& Søren S. A.}

\lstset{mathescape = true}
\usepackage{amsthm}
\begin{document}
\chapter*{Selfstudy 1}
\section*{Exercise 1}
\begin{lstlisting}
Alg(S[1...n]):S'[1...n]
     return Sort(S) // minimum processing time
\end{lstlisting}

\begin{proof}
For our $S'$ we have 
$$p_1 \leq p_2 ... p_{n-1} \leq p_n$$


The average completiontime of this order we call $k$ and is calculated as:
$$\frac{n*p_1 + (n-1)*p_2 + ... 1 * p_n}{n}$$. We swap $a_b$ and $a_c$ where we know that $p_b \leq p_c$ and $b < c \leq n$.

The resulting average completion time will be 
$$k + \frac{(n-c +1)*p_c}{n} - \frac{(n-b +1)*p_b}{n}$$

And due to $b < c \leq n$ and $p_b \leq p_c$ we have that:
$$\frac{(n-c +1)*p_c}{n} - \frac{(n-b +1)*p_b}{n} \geq 0$$

So a swap would not lower the average completion time.
\end{proof}

\section*{Exercise 2}
For a simple bruteforce algorithm for each edge you have the choice of choosing the edge or not choosing it, with $n$ edges we would have the running time $O(2^n)$.

\begin{lstlisting}
Alg(G,s,t)
     if s = t
          T[s,t] $\leftarrow$ 0
     else if T[s,t] = #
          max $\leftarrow$ $-\infty$
          for each e in G.E where e = (s,a)
               if(Alg(G,a,t)+e.w > max)
                    chosen_e $\leftarrow$ e                    
                    max $\leftarrow$ Alg(G,a,t)+e.w
          T[s,t] $\leftarrow$ max, chosen_e
     return T[s,t]
\end{lstlisting}

The subproblem graph would be the current node chosen and all paths from this node to any other node in the graph.

The path is memoized in the table in conjunction with the cost, and can be traversed in linear time.

Each edge and vertice will be visited maximum once, so the running time is be $O(|G.V| + |G.E|)$

\section*{Exercise 3}
If we first sort each point $P_i$ according to their x value. We can for each point choose one of two choices, either the point is in the upper or the lower part of the path, which then leads to two subproblems.
An algorithm can be seen hereafter, where u and $l$ corresponds to the indexes of the upper and lower part.
\begin{lstlisting}
B(u,$l$)
  if T[u,$l$] = #
    if u = n
  	  T[u,$l$] $\leftarrow$ d($P_l$,$P_n$) 	
    else if $l$ = n
  	  T[u,$l$] $\leftarrow$ d($P_u$,$P_n$)
    else if u $\geq$ $l$
      T[u,$l$] $\leftarrow$ max(B(u+1, $l$) + d($P_u$, $P_u+1$),B(u,u+1) + d($P_l$,$P_u+1$))
    else
      T[u,l] $\leftarrow$ max(B($l$+1, $l$) + d($P_u$, $P_l+1$), B(u,$l$+1) + d($P_l$,$P_l+1$))
  return T[u,$l$]
\end{lstlisting}

And has the required running time.
\end{document}