\documentclass[12pt,a4paper]{report}
\usepackage[utf8]{inputenc}
\usepackage{amsmath}
\usepackage{amsfonts}
\usepackage{amssymb}
\usepackage{graphicx}
\usepackage{listings}
\usepackage{fancyhdr}
\usepackage{parskip}
\usepackage{rotating}
\usepackage{caption}
\usepackage{subcaption}
\usepackage{mathtools}
\DeclarePairedDelimiter{\floor}{\lfloor}{\rfloor}

\pagestyle{fancy}
\chead{2.1.10 - sw608f14 - Daniel S. F., Lars A, Mathias W. P. \& Søren S. A.}

\lstset{mathescape = true}
\usepackage{amsthm}

\title{Selstudy 7 - sw608f14 - group room 2.1.10}
\author{Daniel S. F., Lars A, Mathias W. P. \& Søren S. A.}
\begin{document}

\maketitle
\chapter*{Selfstudy 7}
\section*{1. -- 27.3-3}
Assuming parallel copy takes care of copying the arrays in parallel.
So, the $B$ is used as a transfer array to swap subarrays.

\begin{lstlisting}
wrapperpartition(A,p)
	pivot <- A[p]
	Allocate B which is an auxiliary array
     p $\leftarrow$ Algo(B,A,0,A.length-1)
     
Algo(B,A,l,r)
	if l <= r
		if A[l] < pivot
			return l+1
		else
			return l
	else
		mid $\leftarrow$ $\lceil \frac{l+r}{2}\rceil$
		res1 $\leftarrow$ spawn Algo(B,A,l,mid-1)
		res2 $\leftarrow$ Algo(B,A,mid,r)
		sync
		parallel copy A[res1...mid-1] to B[mid...res2]
			     and  A[mid...res2 to B[res1...mid-1]
		then 
		parallel copy B[res1...res2] to A[res1...res2]
\end{lstlisting}

The work is then:
$$W(n) = 2*W(n/2) + O(n) = O(n*lg(n))$$

The span is then
$$S(n) = S(n/2) + O(lg(n)) = O(lg^2(n))$$

So the parallelism is $O(\frac{n}{lg(n)})$

\section*{2.}
So, before splitting on y we have to partition with respect to x.
The algorithm in exercise 1 will keep the order, so we can split and not destroy order.
Before splitting on x we have to partition with respect to y, but not for the first phase, as it is already sorted.
So we only need the presort, as the partition algorithm will keep the order.

The work will be:
$$W(n) = 2* W(n/2) + O(n*lg(n))$$

The span will be:
$$S(n) = S(n/2) + O(lg^2(n)) = lg^3(n)$$

The parallelism is then:
$$\frac{n}{lg^2(n)}$$

\section*{3. -- 35-4}
We did not get to finish this.

\end{document}