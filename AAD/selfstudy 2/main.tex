\documentclass[12pt,a4paper]{report}
\usepackage[utf8]{inputenc}
\usepackage{amsmath}
\usepackage{amsfonts}
\usepackage{amssymb}
\usepackage{graphicx}
\usepackage{listings}
\usepackage{fancyhdr}
\usepackage{parskip}
\usepackage{rotating}
\pagestyle{fancy}
\chead{2.1.10 - sw608f14 - Daniel S. F., Lars A, Mathias W. P. \& Søren S. A.}

\lstset{mathescape = true}
\usepackage{amsthm}
\begin{document}
\section*{Selfstudy 2}
\section*{Exam Exercise 1}
\subsection*{1.}

For the 2nd second, the choice is whether to fire or wait. The remainder of the optimal schedule after that is then the remaining sequence $x_3,...x_n$ with either an $l-k$ of $1$ or $3$.

If the cannon was activated during the 2nd second, it would be the sum of the kills gotten in the 2nd second $min(x_2,f(2))$ and the kills gotten for the remaining $n-2$ seconds with $j=2$.

If the cannon was not activated during the 2nd second, it would be the kills gotten for the remaining $n-2$ seconds with $j=0$.

For the same questions if the cannon was also activated in the first second:

If the cannon was activated during the 2nd second, it would be the sum of the kills gotten in the 1st second $min(x_1,f(1))$, the 2nd second with $min(x_2,f(1))$ and the kills gotten for the remaining $n-2$ seconds with $j=2$.

If the cannon was not activated during the 2nd second, it would be the sum of the kills gotten in the 1st second $min(x_1,f(1))$ and for the remaining $n-2$ seconds with $j=1$.

\subsection*{2.}
A counter example is:
\begin{tabular}{|c|c|c|c|c|}
\hline 
i & 1 & 2 & 3 & 4 \\ 
\hline 
$x_i$ & 1 & 10 & 10 & 1 \\ 
\hline 
$f(i)$ & 1 & 9 & 6 & 9 \\ 
\hline 
\end{tabular}

The algorithm would only shoot at the third and fourth second, whilst the correct answer would be to shoot on the second, third and fourth second.

So the algorithm would eliminate $7$ robots, whilst the correct solution would eliminate $11$ robots.

\subsection*{3.}
The problem looks like
$$D(i,A[1...n],k) =  \begin{cases}
min(A[i],f(i-k))   & i = n \\
max(D(i+1,A,k), min(A[i],f(i-k)) + D(i+1,A,i)) & i < n
\end{cases}$$

Here is the algorithm, where $f(i)$ is the kill potential function and T is a table used for memoization. T is assumed initialized to \#'s and the first call will be with $D(1,A,0)$.
\begin{lstlisting}
D(i,A[1...n],k)
   if T(i,k) = # 
      if i = n
         T(i,k) $\leftarrow$ min(A[i],f(i-k))
      else if i < n
         T(i,k) $\leftarrow$ max(D(i+1,A,k), min(A[i], f(i-k)) + D(i+1,A,i))
   return T(i,k)
\end{lstlisting}

The printing function is as follows.
\begin{lstlisting}
Printer(T[1...n,0...n-1])
   int i $\leftarrow$ 1
   int k $\leftarrow$ 0
   while i $\leq$ n and k $\leq$ n-1
      if(T(i,k) = T(i+1,k)
         i $\leftarrow$ i+1
      else
         k $\leftarrow$ i
         print(i)
         i $\leftarrow$ i + 1
\end{lstlisting}

\subsection*{4.}
The worst case running time of the algorithm is $O(n^2)$, as that is the size of $T$.

\section*{Exercise 2}
\subsection*{1.}
If we invite $P$ we cannot invite $V_1$, $V_2$ and $V_3$. This leaves us with $S_1$, $S_2$ and $S_3$, and gives a total friendliness rating of $5 + (3 + 5 + 2) = 15$.

If we do not invite $P$ we can invite $V_1$, $V_2$, $V_3$, $S_1$, $S_2$, $S_3$.
here the optimal solution is $V_1$, $S_2$, $S_3$, $V_3$ giving a total friendliness rating of $6 + (5 + 2) + 3 = 16$

\subsection*{2.}
\newpage

\begin{sideways}
\parbox{1.1\textheight}{
$$F(n:node,x \in {0,1}) = \begin{cases}
     n.Value  & n.leftMC = null \land n.rightS = null\\
     n.Value + F(n.rightS,0) & n.leftMC = null \land n.rightS \neq null \land x =0\\
     F(n.rightS,1) & n.leftMC = null \land n.rightS \neq null \land x = 1
     \\
     max(n.Value + F(n.leftMC,1), F(n.leftMC,0)) & n.leftMC \neq null
\end{cases}$$}
\end{sideways}
\newpage

Where $x$ indicates whether node n has been chosen ($1$) or not chosen ($0$).
\begin{itemize}
\item Case 1 corresponds to a leaf node, which is a trivial case where you are the last leaf in which case you just return $n.Value$
\item Case 2 is when you are a leaf but you have a sibling, so also a trivial case, but you consider your siblings, with $n$ not having been picked.
\item Case 3 is same as 2 but with $n$ having been picked, thus only propagation of your rightsiblings value.
\item Case 4 is where you actually make your choice on whether to pick $n$ or not.
\end{itemize}

Here is the pseudocode for it:

\begin{lstlisting}[caption=pseudocode for exercise 2 part 1]
F(n:node, x $\in {0,1}$
   if T(n,x) = #
      if n.leftMC = null
         if n.rightS = null
            T(n,x) $\leftarrow$ n.Value
         else if x = 0
            T(n,x) $\leftarrow$ n.Value + F(n.rightS, 0)
         else if x = 1
            T(n,x) $\leftarrow$ F(n.rightS, 1)
      else
         T(n,x) $\leftarrow$ max(n.Value + F(n.leftMC,1), F(n.leftMC,0))
   return T(n,x)
\end{lstlisting}
To print the choices is then simply a matter of recording the choice in the max part of the code.

\subsection*{3.}
The worst case running time is then $O(|V|)$.

\subsection*{4.}
The greedy algorithm would here be to simply pick the employee, skip its children in the tree and pick its grand children and so forth.

The greedy choice is then  to pick the node you are at if your parent is not taken.
The subproblem will then be the part of the tree that is the children of the node considered.

\subsection*{5.}
The running time of the greedy algorithm would also be $O(|V|)$. For exact running time, the greedy algorithm is faster, as it will visit the node once, where the dynamic algorithm will visit the node twice.
\end{document}